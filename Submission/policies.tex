In this section, we define a number of seeding policies. In our analysis, we examine the model under two regimes: budgeted and unbudgeted. In the budgeted regime, we assume we can only pick one starting seed. In the unbudgeted regime we assume we can choose as many starting seed nodes as needed. Note that because we consider $G_\theta$, which consists of the single largest cluster, that a single seed can theoretically reach all nodes within the cluster. For each of the policies defined below, we define the policy under both the budgeted and unbudgeted regime. 

\subsection{Random} 
\textit{Budgeted:}
This policy simply choses at random a seed node from the set of positive target nodes $V_\theta^+$ and serves as the baseline policy with which to compare other heuristics.

\textit{Unbudgeted:}
The policy chooses at random $(1-p) \cdot |V_\theta^+|$ seed nodes from $V_\theta^+$. The intuition behind choosing $(1-p) \cdot |V_\theta^+|$ seed nodes is that the propagation probability $p$ captures how connected the network will be in hindsight when imagining that we flip a coin for each edge to determine if the edge propagates or not in the simulation. Thus, the more connected the network is the fewer seed nodes we will need. Hence, we take $(1-p)$ fraction of the positive target set cardinality. 

\subsection{Degree Centrality} 
\textit{Budgeted:}
This policy chooses from $V_\theta^+$ the node which has the highest degree when considering only other positive target nodes. More specifically, the policy chooses the node within $V_\theta^+$ which has the most number of neighbors also belonging to $V_\theta^+$. Formally, the seed node $v^*$ chosen can be defined as follows: 

\begin{equation}
	v^* = \text{arg}\,\max\limits_{v_i}\  \sum_{v_j \in N(v_i)} 1_{v_j \in V_\theta^+}
\end{equation}

The intuition behind this policy is that this node will have the highest initial propagation rate to other positive target nodes.

\textit{Unbudgeted:}
In the unbudgeted regime, the same algorithm is used, except we choose the top $(1-p) \cdot |V_\theta^+|$ seed nodes using the equation described above.

\subsection{Farthest from Exterior}
\textit{Budgeted:}
This policy is based on the intuition that we want to avoid the "bad" negative nodes. Thus the policy first calculates the expected minimum-distance between all pairs of points within $G_\theta$, and chooses the seed node to be the node which has the largest average expected distance to negative nodes. We use the term "expected distance" because we incorporate into our calculations the probability of reaching a node via the shortest path. Specifically, if the shortest-path to a negative node is 3 (3 edges away), then the expected distance is defined to be $\frac{1}{p^3}$ to reflect the probabilities of each edge. Formally if we define $d_{i,j}$ to be the distance between $v_i$ and $v_j$, then the seed node $v^*$ is chosen as follows:
\\ \\
\begin{equation}
\text{arg}\,\max\limits_{v_i}\  \frac{1}{|V_{\theta}^-|} \sum_{v_j \in V_\theta^-} \frac{1}{p^{d_{i,j}}}
\end{equation}

\textit{Unbudgeted:}
In the unbudgeted regime, the same algorithm is used, except we choose the top $(1-p) \cdot |V_\theta^+|$ seed nodes using the equation described in the budgeted regime.

\subsection{Near-Far} 
\textit{Budgeted:}
This policy combines our desire to be far from negative target nodes and desire to be near positive target nodes. Thus, using the same expected minimum-distance between all points, the policy calculates for each node the sum of the distances to the good nodes minus the sum of the distances to the bad nodes, and chooses the node with the largest value to be the seed node. The seed node $v^*$ is chosen as follows:
\\ \\
\begin{equation}
\text{arg}\,\max\limits_{v_i}\  \sum_{v_j \in V_\theta^+} \frac{1}{p^{d_{i,j}}} -  \sum_{v-j \in V_{\theta}^-} \frac{1}{p^{d_{i,j}}}
\end{equation}

\textit{Unbudgeted:}
In the unbudgeted regime, the same algorithm is used except we include all nodes which are more positive than negative. Intuitively, we can imagine the algorithm as considering all the positive and negative nodes and weighing each by the inverse of its distance. Thus the set of nodes $S$ can be defined as follows:

\begin{equation}
S = \{ v_i |  \sum_{v_j \in V_\theta^+} \frac{1}{p^{d_{i,j}}} -  \sum_{v-j \in V_{\theta}^-} \frac{1}{p^{d_{i,j}}} > 0 \}
\end{equation}

\subsection{Betweenness Centrality}
\textit{Budgeted:}
This policy considers the betweenness centrality of the positive target nodes. More specifically, it calculates the betweenness measure of each node when considering only the positive target nodes in $V_\theta^+$ and chooses the node with highest betweenness centrality. Thus the seed node $v^*$ is calculated as folllows:

\begin{equation}
	\text{arg}\,\max\limits_{v_i}\  \sum_{v_j \neq v_k \neq v_i  \in V_\theta^+} \frac{\sigma_{j,k}(i)}{\sigma_{j,k}}
\end{equation}

where $\sigma_{j,k}$ is the number of shortest-paths between $v_j$ and $v_k$, and $\sigma_{j,k}(i)$ is the number of those shortest-paths which pass through $v_i$.

\textit{Unbudgeted:}
In the unbudgeted regime, the same algorithm is used except similar to before, we choose the top $(1-p) \cdot |V_\theta^+|$ seed nodes using the equation described in the budgeted regime.





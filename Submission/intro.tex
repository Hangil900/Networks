Social networks serve a fundamental role in the spread of ideas throughout the world. With the increased connectivity of the world, social networks have become an ever increasing important aspect of how information spreads. For example, when a new phone app such as Instagram is introduced to the early adopters, these adopters continue to spread the app via the network. How these ideas and information is spread, how rapidly they are spread and to whom they are spread are vital questions surrounding the analysis of social networks. Particularly, imagine the setting where the nodes in a social network are people to whom we want to introduce a new product. These people each have a liking/disliking to the product, and each have a set of neighbors whom they interact with, and may or may not spread the product to. In some settings one might want to maximize the spread of the product throughout the network. In other cases; however, one might want to mitigate for over-exposure and minimize the number of people who are introduced to the product and dislike it. A natural problem which rises from both these settings is how to choose the initial set of early adopters, to whom the product is introduced, such that the desired outcome is achieved.

In this paper, we examine different \textit{seed policies} to choose the initial set of seed adopters to both maximize the spread of influence throughout a social network while also mitigating for over-exposure. More precisely, given a set of positive targets we want to reach, and a set of negative targets we want to avoid, we examine how to choose the initial seed set to introduce a product to, such that the product is spread to positive targets and not spread to negative targets. Further, we consider a stochastic network where each edge has a propagation probability, in the sense that user $A$, who has liked the introduced product, has a certain probability of introducing the product to its neighbor $B$. The design of these policies involve trade-offs between \textit{efficiency} and \textit{simplicity}. Below, we explain what we mean by these two terms:

\noindent \textbf{Efficiency.} The main goal of \emph{seed policies} is to introduce the product to the positve target set and avoid introducing the product to the negative target set. We explain in subsequent sections a performance metric which measures how well a policy achieves this goal and thus how efficient the policy is.

\noindent \textbf{Simplicity. } While efficiency is a highly-desirable aspect of any policy, so is simplicity. If the calculations required to find the seed set is simple, it provides practical benefits for the user of the policy introducing the product. Particularly, if the network in consideration is extremely large, or if information of the network is not perfect, complex policies may no longer become feasible, emphasizing the desire for simple policies. 

We show that... WHAT DO WE SHOW!!


\subsection{Related Work}\label{sec:rel_work}

Include related work: what motivated this.


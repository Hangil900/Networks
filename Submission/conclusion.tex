We have shown several heuristics that vary in the level of knowledge and computing resources available and their effectiveness in maximizing positively activated nodes while minimizing the negatively activated nodes. We see a clear tradeoff between the simplicity and the efficiency of the algorithms. Simpler algorithms that require less information about the network or are less computationally complex, such as Degree policy are easier to compute but result in poor payoff, whereas more complex algorithms, such as Near-Far, which requires us to compute the pairwise distances for all pairs of nodes, perform well across all conditions. However, it is worth noticing that a simpler Degree algorithm only performs slightly worse in most cases and returns a reasonable seed set even when compared to Near-Far. While having sacrificed performance slightly, Degree algorithm may be an attractive choice especially when the size of the network calls for a heavy computation.

Another interesting observation that we can make is that the in unbudgeted case in which we can choose any number of seed agents, we do not see a significant increase in the eventual payoff. One would expect that increasing the size of the seed set would enable us to select a seed set that results in a much higher payoff closer to the optimal payoff of 1. However, we observe that the policy selection contributes more to the performance, as it can be see with the superior performance of Near-Far.

